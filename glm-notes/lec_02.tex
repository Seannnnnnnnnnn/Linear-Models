\lesson{2}{28 Feb 2023}{}

\section{Fitting and Assessing GLMs}
In week 2 of the course, we cover the procedure for fitting GLMs, followed by assessing their goodness of fit. The lectures cover some interesting asymptotic properties of the distribution of fitted parameters, and fit-assessment statistics. Covering asymptotic distributions rigourously however is considered out of scope.

\subsection{Fitting GLMs}
Having specified a GLM with design vector $z$, we will express our data set as $(y_i, z_i(x_i)))$ for $i=1,\dots,n$. In matrix-vector form, we can write these as $y$ and $Z$ where $Z = (z_1, \dots, z_n)^T\in\mathbb{R}^{n\times p}$. \\
\\
The regression coefficients of the model $\beta$ are typically found via maximum likelihood estimation (MLE). Deriving MLE for GLMs is not as straightforward, as a root-solving routine is typically required to find the roots of $\ell^\prime(\beta)$.

\begin{definition}[Score Equation]
    
\end{definition}


\subsection{Assessing Model Fit}


\subsection{Appendix of Algebraic Results}
