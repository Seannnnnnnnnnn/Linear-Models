\lesson{1}{28 Feb 2023}{}

\section{Generalised Linear Models}


Generalised Linear Models (GLMs) are extension of the standard linear model $y_i = \beta x_i + \varepsilon_i$ when the response variable $y_i$ is not normal-like. The most common example of where this occurs is when $y_i$ takes for the form of \textit{count data} like the number of infectious cases for a given day for a given disease, or when $y_i$ takes the form of a \textit{probability}, like the probability that an individual will attend the beach given weather and various other covariates. \\
\\
In week one of this course, we define notation, what a GLM is, and some fundamental properties.

\begin{definition}[Generalised Linear Model] Given covariate-response pairs $(y_i, x_i)$ a generalised linear model is defined by specifying three things:
\begin{enumerate}
    \item \textbf{Design Vector}: This is vector of covariates $z_i(x_i)$. A typical choice is the addition of a regression coefficient $z_i(x_i) = (1, x_{i1}, \dots, x_{ip})$. 

    \item \textbf{Response Distribution}: One must define the distribution of $y_i | x_i$. In particular, the response distribution must be a member of the exponential family of distributions. 
    
    \item \textbf{Response/Link Function}: One must specify the mean of $y_i | x_i$ as a linear function of regression coefficients $\beta$. That is, one must provice a \textit{response function} $h$ such that 
    $$\mu_i = h(\eta_i) = h(z_i^T\beta)$$
    Equivalently, one can define $g = h^{-1}$ which is known as the \textit{link function}.
\end{enumerate}
\end{definition}
\vspace{10pt}
In these notes we will also be working with \textit{grouped} data. This occurs when one or more of the covariates belong to a discrete, categorical format. Instead of expressing the data set $(y_i, x_i)$ for $i=1,\dots,n$, we can truncate the data set, by grouping the samples around groups that share the same categorial combinations. When working with grouped data, we will express our data set as $(\Bar{y_i}, x_i), i=1,\dots,g$ to signify $g$ groups, and $\Bar{y_i}$ the average of responses across group $i$.

\newpage
\subsection{Exponential Family}

\begin{definition}[Exponential Family]
    We say that a random variable $Y$ is a member of the exponential family the probability density of $Y = y$ can be expressed in the following form
    $$f_Y(y|\theta, \phi, \omega) = \exp\left(\frac{y\theta - b(\theta)}{\phi}\omega + c(y,\phi, \omega)\right)$$
    where $\omega=1$ in general, unless when working with grouped data.
\end{definition}
\begin{definition}[Canonical / Natural Link Function]
    Recall in specifying a GLM, one must specify a response function $h$ such that 
    $$E(y_i | x_i) = \mu_i = h(\eta_i) = h(z_i^T\beta)$$
    Once an exponential family distribution has been chosen for $y_i | x_i$, the choice of setting $h(\cdot) = b^\prime(\cdot)$ leads to the \textit{natural response function}. Equivalently, setting 
    $$\theta(\mu_i) = b^{-1}(\mu_i) = \eta_i = z_i^T\beta$$ leads to the \textit{canconial link function}.
\end{definition}
\begin{note} Use of the natural link function leads to variuos convenient statistical properties.
    
\end{note}
\begin{theorem}
    For a given GLM defined on dataset $(y_i, x_i)$ with specified exponential family for the response variable $y_i | x_i$ one can prove that \begin{itemize}
        $$E(y_i | x_i) = \mu_i = b^{\prime}(\theta_i)$$
        and
        $$var(y_i | x_i) = \frac{\phi}{\omega}v(\mu_i)$$
        where $v(\mu_i) := b^{\prime\prime}(\theta)$
    \end{itemize}
\end{theorem}

\begin{remark}[Relationship between mean and variance in GLMs]
    An important consequence of Theorem 1 is that the mean and variance of the GLM are related, and cannot be modelled separately. To see this implication, note that the mean is a function of $\theta_i$, which must be specified when defining the GLM. Subsequently, the second equation is also a function of $\theta$.
\end{remark}

\newpage
\subsection{Example of GLMs}
    

\newpage
\subsection{Notes on Data}


\subsubsection*{Coding Categorical Covariates}

As in common linear regression, qualitative covariates have to be coded ap- propriately. A categorical covariate (or factor) $x$ with $x$ possible categories $1,\dots,k$ will generally be coded by a "dummy vector" with $q = k - 1$ components $x^{(1)},\dots,x^{(q)}$. There are two main flavours of data encoding:

\begin{definition}[Dummy Coding]
    If Categorical covaraiates are 0-1 dummy coded then $x^{(j)}$ for $j=1,\dots,q$ is defined by
    \[ x^{(j)} = \begin{cases} 
      1 & \text{if category $j$ is observed} \\
      0 & \text{otherwise}
   \end{cases}
    \]
    If the $k$th category , the \textit{reference category}, is observed then $x$ is the zero vector
\end{definition}

\begin{definition}[Effect Coding]
    If Categorical covariates are effect coded then $x^{(j)}$ for $j=1,\dots,q$ is defined by
    \[ x^{(j)} = \begin{cases} 
      1 & \text{if category $j$ is observed} \\
      -1 & \text{if category $k$ is observed} \\
      0 & \text{otherwise}
    \end{cases}
    \]
    If the $k$th category , the \textit{reference category}, is observed then $x$ is the vector $(-1,\dots,-1)$
\end{definition}

\subsubsection*{Grouped Data}