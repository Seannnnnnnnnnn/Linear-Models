\documentclass[nocolor]{report}
\input{../preamble.tex}


\begin{document}
    \newpage
    \subsection*{\hspace{10mm} MAST90084 Statistical Modelling. Assignment 1.}
\begin{center}
    Sean Conlon, 1298668 \\
    sconlon@student.unimelb.edu.au
\end{center}


% problem 1 
\begin{ex}[Question 1a] A Poisson sampling model for the contingency table assumes that the $n_{ij}$’s are independently distributed with
$$n_{ij}\sim Poi(\mu_{ij})$$
Where $\mu_{ij}$ denotes the Poisson mean for the cell count $n_{ij}$. \\
\\
Derive the conditional joint distribution of $\{n_{ij}\}_{(i,j)\in\{1,\dots,I\}\times\{1,\dots,J\}}$ given $n$ where $n:=\sum_{1\leq i\leq I, 1\leq j\leq J}n_{ij}$. Identify the name of this distribution, and explcisity state what its parameter values are in terms of the $\mu_{ij}$ and $n$.
\end{ex}
\vspace{-10pt}
\begin{soln}
    
\end{soln}
\begin{proof}

\end{proof}

\begin{ex}[Question 1b] For a $2\times 2$ contingency table the quantity $\frac{\mu_{11}/\mu_{12}}{\mu_{21}/\mu_{22}}$, also known as the \textit{odds ratio}, measures the association between $X$ and $Y$. What should be the values of the odds ratio of $X$ and $Y$ are independant and why?
\end{ex}
\vspace{-10pt}
\begin{soln}
    The \textit{odds ratio} for two independent events $X$ and $Y$ would be 1. We can prove this formally as follows:
\end{soln}
\begin{proof}

\end{proof}


\begin{ex}[Question 3a] When the parameter $\kappa$ is fixed, prove that the Negative Binomial distribution belongs to the exponential dispurtion model. Idenfity the natural parameter $\theta$ and the dispersion paramerte $\phi$ in terms of $\mu$ and $\kappa$ where appropriate, and identify $b(\cdot)$ as a function of $\theta$. Assume $\omega = 1$.
\end{ex}
\begin{proof}
    We begin by noting that 
    \begin{align*}
        \frac{\kappa^\kappa, \mu^y}{(\mu+\kappa)^{\kappa + y}} = \left(\frac{\kappa}{\mu+\kappa}\right)^\kappa \left(\frac{\mu}{\mu+\kappa}\right)^y
    \end{align*}
\end{proof}

\begin{ex}[Question 3b] 
\end{ex}
\vspace{-10pt}
\begin{soln}
    The \textit{odds ratio} for two independent events $X$ and $Y$ would be 1. We can prove this formally as follows:
\end{soln}




\end{document}