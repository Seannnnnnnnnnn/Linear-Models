\documentclass[nocolor]{report}
% basics
\usepackage[utf8]{inputenc}
\usepackage[T1]{fontenc}
\usepackage{textcomp}
% \usepackage[dutch]{babel}
\usepackage{url}
% \usepackage{hyperref}
% \hypersetup{
%     colorlinks,
%     linkcolor={black},
%     citecolor={black},
%     urlcolor={blue!80!black}
% }
\usepackage{graphicx}
\usepackage{float}
\usepackage{booktabs}
\usepackage{enumitem}
% \usepackage{parskip}
\usepackage{emptypage}
\usepackage{subcaption}
\usepackage{multicol}
\usepackage[usenames,dvipsnames]{xcolor}

% \usepackage{cmbright}


\usepackage{amsmath, amsfonts, mathtools, amsthm, amssymb}
\usepackage{mathrsfs}
\usepackage{cancel}
\usepackage{bm}
\newcommand\N{\ensuremath{\mathbb{N}}}
\newcommand\R{\ensuremath{\mathbb{R}}}
\newcommand\Z{\ensuremath{\mathbb{Z}}}
\renewcommand\O{\ensuremath{\emptyset}}
\newcommand\Q{\ensuremath{\mathbb{Q}}}
\newcommand\C{\ensuremath{\mathbb{C}}}
\DeclareMathOperator{\sgn}{sgn}
\usepackage{systeme}
\let\svlim\lim\def\lim{\svlim\limits}
\let\implies\Rightarrow
\let\impliedby\Leftarrow
\let\iff\Leftrightarrow
\let\epsilon\varepsilon
\usepackage{stmaryrd} % for \lightning
\newcommand\contra{\scalebox{1.1}{$\lightning$}}
% \let\phi\varphi





% correct
\definecolor{correct}{HTML}{009900}
\newcommand\correct[2]{\ensuremath{\:}{\color{red}{#1}}\ensuremath{\to }{\color{correct}{#2}}\ensuremath{\:}}
\newcommand\green[1]{{\color{correct}{#1}}}



% horizontal rule
\newcommand\hr{
    \noindent\rule[0.5ex]{\linewidth}{0.5pt}
}


% hide parts
\newcommand\hide[1]{}



% si unitx
\usepackage{siunitx}
\sisetup{locale = FR}
% \renewcommand\vec[1]{\mathbf{#1}}
\newcommand\mat[1]{\mathbf{#1}}


% tikz
\usepackage{tikz}
\usepackage{tikz-cd}
\usetikzlibrary{intersections, angles, quotes, calc, positioning}
\usetikzlibrary{arrows.meta}
\usepackage{pgfplots}
\pgfplotsset{compat=1.13}


\tikzset{
    force/.style={thick, {Circle[length=2pt]}-stealth, shorten <=-1pt}
}

% theorems
\makeatother
\usepackage{thmtools}
\usepackage[framemethod=TikZ]{mdframed}
\mdfsetup{skipabove=1em,skipbelow=0em}


\theoremstyle{definition}

\declaretheoremstyle[
    headfont=\bfseries\sffamily\color{ForestGreen!70!black}, bodyfont=\normalfont,
    mdframed={
        linewidth=2pt,
        rightline=false, topline=false, bottomline=false,
        linecolor=ForestGreen, backgroundcolor=ForestGreen!5,
    }
]{thmgreenbox}

\declaretheoremstyle[
    headfont=\bfseries\sffamily\color{NavyBlue!70!black}, bodyfont=\normalfont,
    mdframed={
        linewidth=2pt,
        rightline=false, topline=false, bottomline=false,
        linecolor=NavyBlue, backgroundcolor=NavyBlue!5,
    }
]{thmbluebox}

\declaretheoremstyle[
    headfont=\bfseries\sffamily\color{NavyBlue!70!black}, bodyfont=\normalfont,
    mdframed={
        linewidth=2pt,
        rightline=false, topline=false, bottomline=false,
        linecolor=NavyBlue
    }
]{thmblueline}

\declaretheoremstyle[
    headfont=\bfseries\sffamily\color{RawSienna!70!black}, bodyfont=\normalfont,
    mdframed={
        linewidth=2pt,
        rightline=false, topline=false, bottomline=false,
        linecolor=RawSienna, backgroundcolor=RawSienna!5,
    }
]{thmredbox}

\declaretheoremstyle[
    headfont=\bfseries\sffamily\color{RawSienna!70!black}, bodyfont=\normalfont,
    numbered=no,
    mdframed={
        linewidth=2pt,
        rightline=false, topline=false, bottomline=false,
        linecolor=RawSienna, backgroundcolor=RawSienna!1,
    },
    qed=\qedsymbol
]{thmproofbox}

\declaretheoremstyle[
    headfont=\bfseries\sffamily\color{RawSienna!70!black}, bodyfont=\normalfont,
    numbered=no,
    mdframed={
        linewidth=2pt,
        rightline=false, topline=false, bottomline=false,
        linecolor=RawSienna, backgroundcolor=RawSienna!1,
    }
]{solnbox}

\declaretheoremstyle[
    headfont=\bfseries\sffamily\color{RawSienna!70!black}, bodyfont=\normalfont,
    numbered=no,
    mdframed={
        linewidth=2pt,
        rightline=false, topline=false, bottomline=false,
        linecolor=RawSienna, backgroundcolor=RawSienna!1,
    },
    qed=\qedsymbol
]{thmproofbox}

\declaretheoremstyle[
    headfont=\bfseries\sffamily\color{NavyBlue!70!black}, bodyfont=\normalfont,
    numbered=no,
    mdframed={
        linewidth=2pt,
        rightline=false, topline=false, bottomline=false,
        linecolor=NavyBlue, backgroundcolor=NavyBlue!1,
    },
]{thmexplanationbox}



% \declaretheoremstyle[headfont=\bfseries\sffamily, bodyfont=\normalfont, mdframed={ nobreak } ]{thmgreenbox}
% \declaretheoremstyle[headfont=\bfseries\sffamily, bodyfont=\normalfont, mdframed={ nobreak } ]{thmredbox}
% \declaretheoremstyle[headfont=\bfseries\sffamily, bodyfont=\normalfont]{thmbluebox}
% \declaretheoremstyle[headfont=\bfseries\sffamily, bodyfont=\normalfont]{thmblueline}
% \declaretheoremstyle[headfont=\bfseries\sffamily, bodyfont=\normalfont, numbered=no, mdframed={ rightline=false, topline=false, bottomline=false, }, qed=\qedsymbol ]{thmproofbox}
% \declaretheoremstyle[headfont=\bfseries\sffamily, bodyfont=\normalfont, numbered=no, mdframed={ nobreak, rightline=false, topline=false, bottomline=false } ]{thmexplanationbox}

\declaretheorem[style=thmgreenbox, name=Definition]{definition}
\declaretheorem[style=thmbluebox, numbered=no, name=Example]{eg}
\declaretheorem[style=thmredbox, name=Proposition]{prop}
\declaretheorem[style=thmredbox, numbered=no, name=Exercise]{ex}
\declaretheorem[style=solnbox, name=Solution]{soln}
\declaretheorem[style=thmredbox, name=Theorem]{theorem}
\declaretheorem[style=thmredbox, name=Lemma]{lemma}
\declaretheorem[style=thmredbox, numbered=no, name=Corollary]{corollary}

\declaretheorem[style=thmproofbox, name=Proof]{replacementproof}
\renewenvironment{proof}[1][\proofname]{\vspace{-10pt}\begin{replacementproof}}{\end{replacementproof}}



\declaretheorem[style=thmexplanationbox, name=Proof]{tmpexplanation}
\newenvironment{explanation}[1][]{\vspace{-10pt}\begin{tmpexplanation}}{\end{tmpexplanation}}

\declaretheorem[style=thmblueline, numbered=no, name=Remark]{remark}
\declaretheorem[style=thmblueline, numbered=no, name=Note]{note}

\newtheorem*{uovt}{UOVT}
\newtheorem*{notation}{Notation}
\newtheorem*{previouslyseen}{As previously seen}
\newtheorem*{problem}{Problem}
\newtheorem*{observe}{Observe}
\newtheorem*{property}{Property}
\newtheorem*{intuition}{Intuition}


\usepackage{etoolbox}
\AtEndEnvironment{vb}{\null\hfill$\diamond$}%
\AtEndEnvironment{intermezzo}{\null\hfill$\diamond$}%
% \AtEndEnvironment{opmerking}{\null\hfill$\diamond$}%

% http://tex.stackexchange.com/questions/22119/how-can-i-change-the-spacing-before-theorems-with-amsthm
\makeatletter
% \def\thm@space@setup{%
%   \thm@preskip=\parskip \thm@postskip=0pt
% }

\newcommand{\oefening}[1]{%
    \def\@oefening{#1}%
    \subsection*{Oefening #1}
}

\newcommand{\suboefening}[1]{%
    \subsubsection*{Oefening \@oefening.#1}
}

\newcommand{\exercise}[1]{%
    \def\@exercise{#1}%
    \subsection*{Exercise #1}
}

\newcommand{\subexercise}[1]{%
    \subsubsection*{Exercise \@exercise.#1}
}


\usepackage{xifthen}

\def\testdateparts#1{\dateparts#1\relax}
\def\dateparts#1 #2 #3 #4 #5\relax{
    \marginpar{\small\textsf{\mbox{#1 #2 #3 #5}}}
}

\def\@lesson{}%
\newcommand{\lesson}[3]{
    \ifthenelse{\isempty{#3}}{%
        \def\@lesson{Lecture #1}%
    }{%
        \def\@lesson{Lecture #1: #3}%
    }%
    \subsection*{\@lesson}
    \testdateparts{#2}
}

% \renewcommand\date[1]{\marginpar{#1}}


% fancy headers
\usepackage{fancyhdr}
\pagestyle{fancy}

% \fancyhead[LE,RO]{Gilles Castel}
\fancyhead[RO,LE]{\@lesson}
\fancyhead[RE,LO]{}
\fancyfoot[LE,RO]{\thepage}
\fancyfoot[C]{\leftmark}

\makeatother




% notes
\usepackage{todonotes}
\usepackage{tcolorbox}

\tcbuselibrary{breakable}
\newenvironment{verbetering}{\begin{tcolorbox}[
    arc=0mm,
    colback=white,
    colframe=green!60!black,
    title=Opmerking,
    fonttitle=\sffamily,
    breakable
]}{\end{tcolorbox}}

\newenvironment{noot}[1]{\begin{tcolorbox}[
    arc=0mm,
    colback=white,
    colframe=white!60!black,
    title=#1,
    fonttitle=\sffamily,
    breakable
]}{\end{tcolorbox}}




% figure support
\usepackage{import}
\usepackage{xifthen}
\pdfminorversion=7
\usepackage{pdfpages}
\usepackage{transparent}
\newcommand{\incfig}[1]{%
    \def\svgwidth{\columnwidth}
    \import{./figures/}{#1.pdf_tex}
}

% %http://tex.stackexchange.com/questions/76273/multiple-pdfs-with-page-group-included-in-a-single-page-warning
\pdfsuppresswarningpagegroup=1


\author{Sean Conlon}


\begin{document}
    \newpage
    \subsection*{\centering MAST90012 Measure Theory. Assignment 3.}
\begin{center}
    Sean Conlon, 1298668 \\
    sconlon@student.unimelb.edu.au
\end{center}


% problem 1 
\begin{ex}[Question 1a] We consider convolution and $L^p$ functions with respect to the Lebesgue measure $m$ on $\mathbb{R}^d$. Let $\phi_1$ be a smooth, non-negative function which is supported on the unit ball $B_1(0)$ and satisfies $\int \phi_1 dm = 1$. For $\sigma>0$ we define the $\phi_\sigma = \sigma^{-d} \phi_1(x/\sigma)$ For $1\leq p <\infty$ and any $f\in L^p(\mathbb{R}^d)$ and $\sigma > 0$ we define
$$f_\sigma (x) = \int_{\mathbb{R}^d} f(y)\phi_\sigma(x-y)dy$$
Now, for all $f\in L^p$ show that $\|f_\sigma\|_{L^p}\leq\|f\|_{L^p}$
\end{ex}
\begin{proof}
    For let $\frac{1}{p} + \frac{1}{q}=1$ and for a general $f,g$ we have 
    \begin{align*}
        |(f*g)(x)| &\leq \int |g(y)f(x-y)|dy \\
                   &\leq \int |g(y)f(x-y)^{\frac{1}{p}}||f(x-y)|^{\frac{1}{q}}dy \\
                   &\leq \left(\int|f(x-y)|dy\right)^{\frac{1}{q}} \left(\int|g(y)|^p |f(x-y)|\right)^{\frac{1}{p}} \quad \text{By Holder} \\
                   &= \|f\|_1^{\frac{1}{q}}  \left(\int|g(y)|^p |f(x-y)|dy\right)^{\frac{1}{p}} 
    \end{align*}        
    Therefore we have 
    \begin{align*}
         \int |(f*g)(x)|^p dx &\leq \|f\|_1^{\frac{p}{q}}  \int \int|g(y)|^p |f(x-y)|dydx \\
         &= \|f\|_1^{\frac{p}{q}}  \int \int|g(y)|^p |f(x-y)|dx dy \quad \text{By Fubini} \\
         &= \|f\|_1^{\frac{p}{q}} \|f\|_1 \|g\|_p^p
    \end{align*}
    Therefore taking the $p^{th}$ root of both sides yields
    $$\|f * g\|_p \leq \|f\|_1 \|g\|_p$$
    We now consider the case of $f_\sigma(x) = (f*\phi_\sigma)(x) = (\phi_\sigma * f)(x)$ to have 
    $$\|f_\sigma\| \leq \|f\|_p \|\phi_\sigma\|_1$$
    Though by our assumptions on $\phi_\sigma$ and $\phi_1$ we have $\|\phi_\sigma\|\leq1 $ and so as claimed $\|f_\sigma\|_p\leq \|f\|_p$ .
\end{proof}

\begin{ex}[Question 1b] Show that for each $f\in L^p(\mathbb{R}^d)$ 
$$\lim_{\sigma \rightarrow 0^+}\|f_\sigma - f\|=0$$
\end{ex}
\begin{proof}
    We consider  
    \begin{align*}
        |f_\sigma(x) - f(x)| &= |(f*\phi_\sigma)(x) - f(x)| \\
        &\leq \int_{\mathbb{R}^d} |f(x-y)-f(x)|\phi_\sigma(y)dy
    \end{align*}
    Now applying Jensen's inequality to the above we have 
    \begin{align*}
         |(f*\phi_\sigma)(x) - f(x)|^p &\leq \left( \int_{\mathbb{R}^d} |f(x-y)-f(x)|\phi_\sigma(y)dy \right)^p  \\
         &\leq  \int_{\mathbb{R}^d} |f(x-y)-f(x)|^p\phi_\sigma(y)dy
    \end{align*}
    Now further integrate over $x$ and apply Fubinis theorem to yield 
    \begin{align*}
        \|f_\sigma - f\|_p^p &\leq  \int_{\mathbb{R}^d}\int_{\mathbb{R}^d} |f(x-y)-f(x)|^p\phi_\sigma(y)dydx \\
        &=   \int_{\mathbb{R}^d}\left(\int_{\mathbb{R}^d} |f(x-y)-f(x)|^p dx\right) \phi_\sigma(y)dy \\
        &= \int_{\mathbb{R}^d}\|f(x-y) -f(x)\|_p^p \phi_\sigma(y)dy
    \end{align*}
    Now, one defines $g(y) := \|f(x-y) - f(x)\|^p_p$ and notes that $g(0) = 0$. Now we may write
    \begin{align*}
        \|f_\sgma - f\|^p_p &\leq \int_{\mathbb{R}^d} g(y)\phi_\sigma(y)dy \\
        &= \int_{\mathbb{R}^d} g(y)\sigma^{-d}\phi\left(\frac{y}{\sigma}\right)dy \\
        &= \int_{\mathbb{R}^d} g(\sigma z)\sigma^{-d}\phi(z)dz \hspace{10mm} z:= \frac{y}{\sigma} 
    \end{align*}
    Hence, as $\sigma\rightarrow0$ we have $g(\sigma z)\phi(z) \rightarrow g(0)\phi(z) = 0$. Moreover, we can show that $g$ is bounded by: 
    $$g(y) = \|f(x-y)-f(x)\|^p_p \leq (\|f(x-y)\|_p + \|f(x)\|_p)^p = (2\|f\|)^p$$
    The boundedness of $g$. Combined with the integrability of $\phi$ mean we may apply the Dominated Convergence Theorem to show that 
    $$\lim_{\sigma \rightarrow 0} \int_{\mathbb{R}^d} g(\sigma z)\phi(z)dz = 0$$
    Which implies the claim and thus completes the proof.
\end{proof}

\newpage
\begin{ex}[Question 2a] 
Let $(X,\mathcal{A})$ be a measurable space and $\mu^*$ be an outer measure on $X$. Define the following outer measure $\mu^+$ as follows. For any $E\subset X$ we have 
$$\mu^+(E) = \inf\left\{ \sum_{j=1}^{\infty}\mu^*(E_j)  \big| E\subseteq\bigcup_{j=1}^{\infty}E_j \quad E_j \text{ is $\mu^*$ Caratheodory measurable}\right\}$$
Show that for all $E\subseteq X$ we have $\mu^*(E)\leq\mu^+(E)$
\end{ex}
\begin{proof}
    Let $E\subseteq X$ such that $E\subseteq \cup_{j=1}^{\infty} E_j$ for $\{E_j\}_{j=1}^{\infty}$ $\mu^*$ Caratheodory measurable in $X$. By the monotonicity and subadditivity of $\mu^*$ one has: 
    \begin{align*}
        \mu^*(E) \leq \mu^*\left( \bigcup_{j=1}^{\infty} E_j\right) \leq \sum_{j=1}^{\infty}\mu^*(E_j)
    \end{align*}
    Now, take $\{E_j\}_{j=1}^{\infty}$ to be the \textit{infimum} over the possible collection of sets with respect the outer-measure $\mu^*$ such that $E\subseteq \cup_{j=1}^{\infty} E_j$. That is: 
    \begin{align*}
         \mu^*(E) \leq \sum_{j=1}^{\infty}\mu^*(E_j) &= \inf\left\{ \sum_{j=1}^{\infty}\mu^*(E_j) \big | E\subseteq\bigcup_{j=1}^{\infty} E_j\right\} \\
         &= \mu^+(E)
    \end{align*}
    Thus, $\mu^*(E)\leq\mu^+(E)$ as claimed. 
\end{proof}
\begin{ex}[Question 2b] 
    Given a set $E\subset X$, show that $\mu^*(E) = \mu^+(E)$ if and only if, there exists a $\mu^*$-measurable set $A$ such that $E\subseteq A$ with $\mu(A) = \mu^*(A)$
\end{ex}
\begin{proof} $(\Rightarrow)$. \\
Suppose $\mu^*(E) = \mu^+(E)$ then 
$$\mu^*(E) =  \inf\left\{ \sum_{j=1}^{\infty}\mu^*(E_j)  \big| E\subseteq\bigcup_{j=1}^{\infty}E_j\right\} \quad (1)$$
Where $\{E_j\}_{j=1}^{\infty}$ are $\mu^*$-Caratheodory measurable. let $\{E_j\}_{j=1}^{\infty}$ be a covering of $E$ such that equality in $(1)$ holds. Moreover, assume that $\{E_{j, i}\}_{i=1}^{\infty}$ is a covering of each $E_j$ such that each $E_{j, i}$ is $\mu^*$-measurable and 
$$\mu^*(E_{j, i}) \leq \sum_{i=1}^{\infty} \mu^*(E_j) +\frac{\varepsilon}{2^j}$$
For arbitrary $\varepsilon>0$. Now, set $A = \cup_{j=1}^{\infty}\cup_{i=1}^{\infty}E_{j, i}$. As $E = \cup_{j=1}^{\infty}E_j$ we therefore have $E\subseteq A$. Moreover, by the monotonicity of $\mu^*$ we have 
\begin{align*}
    \mu^*(E) \leq \mu^*(A) &\leq \sum_{j=1}^{\infty}\sum_{i=1}^{\infty}\mu^*(E_{j,i}) \\
    &\leq \sum_{j=1}^{\infty} \mu^*(E_j) +\frac{\varepsilon}{2^j} \\
    &= \varepsilon + \mu^*(E)
\end{align*}
Where the equality in the final line comes from choice of $\{E_j\}$ such that $(1)$ holds. Taking $\varepsilon\rightarrow0^+$ we conclude that $\mu^*(E) = \mu^*(A)$. Moreover, as $A$ is chosen to be a countable union of $\mu^*$-Caratheodory measurable sets, we conclude that $A$ is itself measurable. By our previous theorem, we can take measure $\mu$ on the measurable space $(X,\mathcal{A})$ by taking $\mathcal{A}$ to be the set of Caratheodory measurable sets and $\mu = \mu^* |_{\mathcal{A}}$ to conclude that $\mu(A) = \mu^*(A)$ as required.  
\end{proof}
\begin{proof}$(\Leftarrow)$. \\
    Let $\mathcal{M}$ be $\sigma$-Algebra formed by the $\mu^*$- measurable subsets of $X$ and thus $\mu = \mu^*|_{\mathcal{M}}$ is a measure. Let $A\in \mathcal{M}$ such that $\mu(A) = \mu^*(E)$ and $E\subseteq A$. \\
    \\
    let $\{A_j\}_{j=1}^{\infty}$ be the infimum over $\mu^*$- measurable subsets of $X$ such that $A\subseteq \cup_{j=1}^{\infty}A_j$. Now define the collection of pairwise-disjoint sets $\{A^\prime_j\}^{\infty}_{j=1}$ by:
    $$A^\prime_1 = A_1 \quad \quad A^\prime_j = A_j \backslash \bigcup_{k=1}^{j-1}A_j$$
    so that $\{A_j^\prime\}$ is also a covering of $A$ with equal measure to $\{A_j\}$. Moreover, one notes that as $A^\prime_j$ is a countable intersection and union of $\mu^*$-measurable sets $\{A_j\}$, then $\{A^\prime_j\}$ is also $\mu^*$-measurable. It now follows 
    \begin{align*}
        \mu^+(E) & \leq \mu^+(A) \hspace{12mm} \text{by monotonicity of } \mu^+ \\
        &= \sum_{j=1}^{\infty}\mu^*(A_j^\prime) \\
        &= \sum_{j=1}^{\infty}\mu(A_j^\prime) \hspace{10mm} \text{as for any } E\in\mathcal{M}, \mu^*(E) = \mu(E) \\
        &= \mu(A) = \mu^*(E)
    \end{align*}
    Hence we have shown $\mu^+(E)\leq\mu^*(E)$, and in combination with exercise 2a, we conclude $\mu^+(E) = \mu^*(E)$.
\end{proof}


\begin{ex}[Question 3a] Consider the measurable space $([0,1],\mathcal{B})$, where $\mathcal{B}$ is the collection of Borel Subsets of $[0,1]$. let $m:\mathcal{B}\rightarrow[0,\infty]$ be the Lebesgue measure and let $c:\mathcal{B}\rightarrow[0,\infty]$ be the counting measure. We consider the diagonal
$$D = \{(x,y)\in[0,1]\times[0,1]: x=y\}$$
Show that $D$ is  Carath´eodory measurable with respect to the canonical $(m\times c)^*$ outer measure.
\end{ex}
\begin{proof}
    Consider the family of sets $D_n$ given by: 
    $$D_n = \bigcup_{k=0}^{n-1}\left[ \frac{k}{n}, \frac{k+1}{n}\right]\times \left[ \frac{k}{n}, \frac{k+1}{n}\right] $$
    Hence, each $D_n$ is a countable union of $(m\times c)^*$-measurable intervals and is therefore, itself measurable. Moreover, 
    $$D = \bigcap_{n=1}^{\infty}D_n$$
    Hence, $D$ is a $R_{\sigma\delta}$ set over measurable sets, and is therefore itself measurable.
\end{proof}
\begin{ex}[Question 3b]
    Let $f$ be the characteristic function of $D$, show that Fubini's theorem fails in this case.
\end{ex}
\begin{proof}
    Let $I = [0,1]$ and $f(x,y) = \chi_D(x,y)$. We consider the integral
    \begin{align*}
        \int_I \int_I f(x,y) dc(y) dm(x) &=  \int_I\left( \int_I f(x,y) dc(y)\right) dm(x)\\
        &=  \int_I\left( \int_I 1_{x}(y)dc(y)\right) dm(x) \\
        &= \int_I 1 dm(x) \\
        &= 1
    \end{align*}
    However, we also have 
    \begin{align}
        \int_{I\times I} \chi_D d(m\times c) = \infty
    \end{align}
    For a proof of $(1)$, suppose that $D$ is coverered by $\{A_k \times B_k\}_{k=1}^{\infty}$. Let $\{A_j\}_{j=1}^{M}$ be a subset of $\{A_k\}$ such that $m(A_j) = 0$. We can now define 
    $$S = \bicup_{j=1}^{M}A_j$$
    So that $m(S) = 0$ and $[0,1]\backslash S$ is uncountable so $c([0,1]\backslash S) = \infty$. Now define the set 
    $$F = \{(x,y):  x=y, x,y\in[0,1]\backslash S\}$$
    By definition $F\subseteq E$, and moreover $\{A_k\times B_k\}\backslash F$ is therefore also a covering of $F$. Let $(x,x)\in F$, then for some $k$ we have $x\in A_k$, $x\in B_k$
    Hence, we may conclude that $[0,1]\backslash S \subseteq \cup_{k=1}^{\infty}B_k$. However, $[0,1]\backslash S$ is uncountable, and thus, there must be at least one $B_{k^\prime}$ in $\{B_k\}$ such that $c(B_{k^\prime}) = \infty$ and thus $\{A_k\times B_k\}$ must also have $(\mu\times c)(A_{k^\prime}\times B_{k^\prime}) = \infty$\\
    \\
    Finally, we note that we have shown in this exercise how Fubinis theorem can fail to hold in the case of one the measures, in this case, the counting measure $c$, failing to be $\sigma$-finite.
\end{proof}


\begin{ex}[Question 4] Prove the Uniqueness of the Jordan Decomposition Theorem. That is, let $\nu$ be a signed measure on the measurable space $(X,\mathcal{A})$. Show that there is a unique pair of mutually singlular measures $\nu^+$, $\nu^-$ on $(X,\mathcal{A})$ such that $\nu = \nu^+ - \nu^-$
\end{ex}
\begin{proof}
    let $\nu = \nu^+ - \nu^-$ be a Jordan Decomposition of $\nu$ and let $(P, N)$ be a Hanh decomposition of $(X,\mathcal{A})$ with respect to signed measure $\nu$. We claim that for any $A\in\mathcal{A}$ we have 
    \begin{align*}
        & \nu^+(A) = \sup_{B\subset A, B\in \mathcal{A}} \nu(B) \\
        &\nu^-(A) =  -\inf_{B\subset A, B\in \mathcal{A}} \nu(B)
    \end{align*}
    To prove the claim, let $B\in\mathcal{A}$ and $B\subset A$ then one have
    \begin{align*}
        \nu(B) &= \nu^+(B) - \nu^-(B) \\
        &\leq \nu^+(B) \\
        &\leq \nu^+(A) \\
        &= \nu(A\cap P) 
    \end{align*}
    We now note by properties of the $\sigma$-Algebra $A\cap P\in\mathcal{A}$ and so by definition of the Jordan decomposition
    $$\nu^+(A) = \nu(A\cap P)\leq \sup_{B\in\mathcal{A}, B\subseteq A}\nu (B)$$
    Hence proving the first claim. The second claim for $\nu^-$ follows a similar proof, 
    \begin{align*}
        -\nu(B) &= \nu^-(B) - \nu^+(B) \\
                &\leq \nu^-(B) \\
                &\leq \nu^-(A)
    \end{align*}
    And with similar argument 
    $$\nu^-(A) = -\nu(A\cap N) \leq - \inf_{B\subset A, B\in \mathcal{A}} \nu(B)$$
    Having proved the claim, the uniquness of the Jordan Decomposition follows easily. Suppose there are two Jordan decompositions of the signed measure $(\nu^+_1, \nu^-_1)$ and $(\nu^+_2, \nu^-_2)$. Then by our claim for any set $A\in \mathcal{A}$
    \begin{align*}
       & \nu_1^+(A) =  \sup_{B\subset A, B\in \mathcal{A}} \nu(B) = \nu_2^+(A) \\
       & \nu_1^-(A) =  -\inf_{B\subset A, B\in \mathcal{A}} \nu(B) = \nu_2^-(A)
    \end{align*}
    So we conclude that the decompositions are equal.
\end{proof}





\end{document}